\documentclass[11pt]{article}

% Use wide margins, but not quite so wide as fullpage.sty
\marginparwidth 0.5in 
\oddsidemargin 0.25in 
\evensidemargin 0.25in 
\marginparsep 0.25in
\topmargin 0.25in 
\textwidth 6in \textheight 8 in
% That's about enough definitions


\begin{document}
\hfill\vbox{\hbox{Delaplaine, Victor}
		\hbox{Cpe 453, Section 01}	
		\hbox{Lab 2}	
		\hbox{\today}}\par

\bigskip
\centerline{\Large\bf Lab 2: Minix Scavenger Hunt}\par
\bigskip 

Hello, for this laboratory I will be using {\LaTeX} for my choice of document prepartion. 
I also have chosen to use {\tt VirtualBox} for hosting Minix 3.1.8 on.

\setcounter{section}{1} % make the first section, section 2(see below)
\section{Logging in}


\subsection{Approach}

I was prompted immedately with:

{\tt\begin{tabbing}
\# login: \\
\end{tabbing}}

For a long time I was typing in my Cal Poly username "vdelapla":

{\tt\begin{tabbing}
\# login: vdelapla\\
\\ Password: *****\\
\end{tabbing}}


\subsection{Problems Encountered}
  Each time I would enter my username I would get a login error.

\subsection{Solutions}
  After getting a lot of "Login incorrect" messages, I realized it was time to 
read the manual. It said that to login with the name "root" and continue with 
the following steps.


\subsection{Lessons Learned}
  In this section I learned to read things more closely.

\setcounter{section}{2} % make the first section, section 2(see below)
\section{Create a user account}


\subsection{Approach}

  The create a user account section consists creating a new 
username and setting a password for that username. 
The user account I created is "victor", its group is "operator", 
and home directory is "/home/victor".

\subsection{Problems Encountered}

  When it was time to create a password for my login name "victor", 
I typed in the shell "passwd" while logged in as root. This prompted 
me to change the password for root:

{\tt\begin{tabbing}
\# passwd\\
Changing the shadow password of root\\
New pasword:\\
\end{tabbing}}

\subsection{Solutions}

   The solution was to exit "passwd" by hitting CTRL-C, 
and then to read the man page for "passwd". In the man page, I came across
an optional argument, that changes the password/set of an existing user. This command
otherwise defaults to the signed-in user.

\subsection{Lessons Learned}
  In this section, I learned to not assume anything about the functionality of a program, 
and always go back to read the man page if im not sure what it does.


\setcounter{section}{3} % make the first section, section 2(see below)
\section{Create a Minix disk image and use it to store data}

\subsection{Approach}

  I needed to create an empty disk image file, 
so I issued the "dd if=/dev/zero of=testFloppy.img bs=1024 count=1440" 
command. Then went on the {\tt Virtual Box} settings and configured 
the "testFloppy.img" to be used for Minix's drive. After coniguring the above, 
I then formated the floppy drive, made the file system, mounted the floppy, 
and finally unmounted it. Below shows the commands after attatching the 
floppy drive in {\tt Virtual Box}.

{\tt\begin{tabbing}
format /dev/fd0 1440\\
mkfs /dev/fd0\\
mkdir /mnt/Floppy\\
mount /dev/fd0 /mnt/Floppy\\
umount /mnt/Floppy
\end{tabbing}}

\subsection{Problems Encountered}

  When, I was trying to un mount the floppy drive using the above 
command. I issued it before making a directory called "Floppy" and got an 
error that said:

{\tt\begin{tabbing}
mount: Can't mount /dev/fd0 on /mnt/Floppy/: No such file or directory
\end{tabbing}}

\subsection{Solutions}

The solution was to create an empty folder named "Floppy" by using "mkdir /mnt/Floppy".

\subsection{Lessons Learned}
  Again, I learned that you cannot assume that a program does something.

		
\setcounter{section}{4} % make the first section, section 2(see below)
\section{Acessing your data from outside Minix}

\subsection{Approach}

  I configured the network for Minix by first using netconf and using the recommened 
network card. After this I shut down the Minix machine, went into VirtualBox settings and selected 
the bridge connection option. Finally, 
after this I restarted Minix looked up its ip address, then ran:

{\tt\begin{tabbing}
scp -P 22 victor@<ip address>:/home/victor/.ashrc .
\end{tabbing}}

  On my local machine that I ran this command I was able to get the file ".ashrc" onto my
current working directory.


\subsection{Problems Encountered}

  I tried to run "scp" as "ssh" where you just provide the <hostname>@<ip address>,
because I thought it was going to give me a shell like "ftp".


\subsection{Solutions}

  I looked at the man page for "scp"; it said that after the target ip address to put the 
file name you want, and where you want to place it.

\subsection{Lessons Learned}
 I have learned that to read the man page, before assumptions are made.
  
  

\end{document}
